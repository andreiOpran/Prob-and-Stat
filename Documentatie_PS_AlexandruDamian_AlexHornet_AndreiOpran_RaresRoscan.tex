\documentclass{article}
\usepackage{graphicx} 
\usepackage{amsmath, amssymb}
\usepackage[a4paper,margin=1in]{geometry} 
\usepackage{parskip} 
\usepackage{xcolor}
\usepackage{listings} 
\usepackage{hyperref}

\title{\textbf{Proiect la Probabilități și Statistică}}
\author{
    \vspace{1cm} \\
    Universitatea din București \\
    Facultatea de Matematică și Informatică \\
    \vspace{0.2cm} \\
    Profesor curs: Lect.dr. Alexandru Amărioarei \\
    Profesor laborator: Andriciuc Alexandra \\
    \vspace{0.5cm} \\
    Alexandru Damian, Alex Horneț, Andrei Opran, Rareș Roșcan \\
    \vspace{0.2cm} \\ 
    Lider: Andrei Opran \\
    Grupa 242 \\
    \vspace{0.2cm} \\ 
    \href{https://github.com/andreiOpran/Prob-and-Stat}{\texttt{GitHub Repository}}
}
\date{}

\lstset{
  language=R,
  basicstyle=\ttfamily\footnotesize,
  keywordstyle=\color{blue},
  stringstyle=\color{red},
  breaklines=true,
  frame=single
}

\begin{document}

\maketitle

\newpage

\section {Problema 1}

\subsection{Subpunctul 1: Generarea punctelor in patratul $[-1,1]^2$}

Fie doua variabile aleatoare independente \( X_1 \) și \( X_2 \) repartizate uniform pe intervalul \( [-1,1] \). Densitățile lor sunt:
\[
f_{X_1}(x_1) = \frac{1}{2}, \quad f_{X_2}(x_2) = \frac{1}{2}, \quad \forall x_1, x_2 \in [-1,1].
\]
Densitatea comuna, deoarece variabilele sunt independente, este:
\[
f_{(X_1,X_2)}(x_1, x_2) = f_{X_1}(x_1) \cdot f_{X_2}(x_2) = \frac{1}{2} \cdot \frac{1}{2} = \frac{1}{4}, \quad \forall (x_1, x_2) \in [-1,1]^2.
\]

\subsubsection*{Cod R utilizat}
\begin{lstlisting}
set.seed(88)
N <- 10000  
X1 <- runif(N, min = -1, max = 1)
X2 <- runif(N, min = -1, max = 1)
plot(X1, X2, main = "Puncte distribuite uniform pe patratul [-1,1]^2",
     xlab = "X1", ylab = "X2", pch = 20, col = rgb(0, 0, 1, 0.5))
abline(h = c(-1, 1), v = c(-1, 1), col = "red", lwd = 2)
\end{lstlisting}

\newpage

\subsubsection*{Reprezentare grafica punctelor generate}

\begin{figure}[h]
    \centering
    \includegraphics[width=1\textwidth]{1_1.png}
    \caption{Puncte distribuite uniform pe pătratul $[-1,1]^2$}
    \label{fig:patrat}
\end{figure}

\newpage

\subsection{Subpunctul 2: Metoda acceptării și respingerii pentru discul unitate}

Pentru a selecta punctele care se află în interiorul discului unitate \( D(1) \), aplicăm metoda acceptării și respingerii.  
Punctele acceptate sunt colorate în albastru, iar cele respinse în roșu.

\subsubsection*{Cod R utilizat}
\begin{lstlisting}
inside_disc <- (X1^2 + X2^2) <= 1

plot(X1[!inside_disc], X2[!inside_disc], col = "red", pch = 20, 
     xlab = "X1", ylab = "X2", main = "Simulare puncte in discul unitate")
points(X1[inside_disc], X2[inside_disc], col = "blue", pch = 20)  

symbols(0, 0, circles = 1, inches = FALSE, add = TRUE, lwd = 2)

legend("topright", legend = c("In interior", "Respins"), col = c("blue", "red"), pch = 20)
\end{lstlisting}

\newpage

\subsubsection*{Reprezentare grafică a punctelor generate}

\begin{figure}[h]
    \centering
    \includegraphics[width=1\textwidth]{1_2.png}
    \caption{Simulare a punctelor în discul unitate folosind metoda acceptării și respingerii}
    \label{fig:disc_acceptare}
\end{figure}

\newpage

\subsection{Subpunctul 3: Calculul mediei aritmetice a distanțelor față de origine}

Calculăm media aritmetică a distanțelor punctelor acceptate față de origine și o comparăm cu valoarea teoretică \( \frac{2}{3} \).

\subsubsection*{Cod R utilizat}
\begin{lstlisting}
distances <- sqrt(X1[inside_disc]^2 + X2[inside_disc]^2)
mean_distance <- mean(distances)
theoretical_mean_distance <- 2/3

cat("Media aritmetica a distantelor punctelor de origine:", mean_distance, "\n")
cat("Media teoretica a distantei:", theoretical_mean_distance, "\n")
\end{lstlisting}

\subsubsection*{Rezultatele obținute}

După rularea codului R, am obținut următoarele rezultate:

\begin{itemize}
    \item \textbf{Media aritmetică a distanțelor punctelor de origine:} 0.6675077
    \item \textbf{Media teoretică a distanței:} 0.6666667
\end{itemize}

\subsection{Subpunctul 4: Determinarea densităților pentru $ R $ și $ \theta $}

Știm că punctele \( (X, Y) \) sunt distribuite uniform în discul unitate \( D(1) \), având densitatea constantă:
\[
f_{(X,Y)}(x,y) = \frac{1}{\pi}, \quad \forall (x,y) \in D(1).
\]

Aplicăm schimbarea de variabilă:
\[
X = R \cos(\theta), \quad Y = R \sin(\theta).
\]

Calculăm Jacobianul transformării:
\[
J = \begin{vmatrix} 
\cos\theta & -R\sin\theta \\
\sin\theta & R\cos\theta 
\end{vmatrix}
= \begin{vmatrix} 
R (\cos^2\theta + \sin^2\theta) 
\end{vmatrix} = R.
\]

Folosind formula de schimbare de variabilă:
\[
f_{(R,\theta)}(r, \theta) = f_{(X,Y)}(x,y) \cdot |J| = \frac{1}{\pi} \cdot R = \frac{r}{\pi}, \quad 0 \leq r \leq 1, \quad 0 \leq \theta \leq 2\pi.
\]

Densitățile marginale sunt:
\[
f_R(r) = \int_0^{2\pi} f_{(R, \theta)}(r, \theta) d\theta = \int_0^{2\pi} \frac{r}{\pi} d\theta = \frac{r}{\pi} \cdot 2\pi = 2r, \quad 0 \leq r \leq 1.
\]

\[
f_\theta(\theta) = \int_0^1 f_{(R, \theta)}(r, \theta) dr = \int_0^1 \frac{r}{\pi} dr = \frac{1}{\pi} \cdot \frac{1}{2} = \frac{1}{2\pi}, \quad 0 \leq \theta \leq 2\pi.
\]


\newpage

\newpage

\subsection{Subpunctul 5: Simularea a $N = 1000$ puncte folosind coordonate polare}

Continuăm cu metoda coordonatelor polare pentru a genera puncte distribuite uniform pe discul unitate \( D(1) \).  
Această metodă presupune:
\begin{itemize}
    \item Alegerea unui unghi aleator \( \theta \sim U(0,2\pi) \).
    \item Alegerea unei raze aleatoare \( R = \sqrt{U} \), unde \( U \sim U(0,1) \).
    \item Calcularea coordonatelor carteziene:
          \[
          X = R \cos(\theta), \quad Y = R \sin(\theta).
          \]
\end{itemize}

\subsubsection*{Cod R utilizat pentru generarea punctelor}
\begin{lstlisting}
##### SUBPUNCTUL 5 #####
cat("\n=============== SUBPUNCTUL 5 ===============\n")
cat("Generam puncte folosind metoda coordonatelor polare\n")

N <- 1000  # Numarul de puncte

# Generam unghiul theta uniform in [0, 2*pi]
theta <- runif(N, min = 0, max = 2*pi)

# Generam raza R folosind transformarea sqrt(U), unde U ~ U(0,1)
R <- sqrt(runif(N, min = 0, max = 1))

# Calculam coordonatele carteziene
X <- R * cos(theta)
Y <- R * sin(theta)

# Plotare puncte
plot(X, Y, main = "Puncte distribuite uniform pe discul unitate (metoda coordonatelor polare)",
     xlab = "X", ylab = "Y", pch = 20, col = rgb(0, 0, 1, 0.5))

# Adaugam conturul cercului
symbols(0, 0, circles = 1, inches = FALSE, add = TRUE, lwd = 2)
\end{lstlisting}

\newpage

\subsubsection*{Reprezentare grafică a punctelor generate}

\begin{figure}[h]
    \centering
    \includegraphics[width=1\textwidth]{1_5.png}
    \caption{Puncte distribuite uniform pe discul unitate folosind coordonate polare}
    \label{fig:disc_polar_1000}
\end{figure}

\newpage

\section{Problema 2 - \href{https://broski-corp.shinyapps.io/Exercitiul-2-PS/}{\texttt{link către aplicație}}}
\vspace{2em}

\subsection*{Descrierea problemei}
\vspace{0.4em}
Aplicația propusă are ca scop oferirea unei platforme prin care utilizatorii pot vizualiza funcțiile de repartiție ale diverselor variabile aleatoare. Obiectivul principal este de a furniza o interfață interactivă ce permite utilizatorilor să selecteze distribuțiile dorite și parametrii aferenți, iar aplicația va genera automat graficele corespunzătoare acestor setări.

\subsection*{Aspecte teoretice care depășesc nivelul cursului}
\vspace{0.4em}
Pentru implementarea aplicației, este necesar să înțelegem conceptul de framework \textbf{Shiny}, care facilitează dezvoltarea de aplicații web interactive utilizând limbajul R. Shiny este structurat în două componente principale:
\begin{itemize}
    \item \textbf{UI (User Interface)}: Aceasta definește structura vizuală a aplicației și elementele interactive, precum butoane, meniuri și casete de selecție.
    \item \textbf{Server}: Această componentă conține logica aplicației, gestionând datele introduse de utilizator și generând rezultatele ce vor fi afișate pe interfața utilizatorului.
\end{itemize}
De asemenea, este esențială utilizarea observer-ului în Shiny pentru actualizarea automată a rezultatelor pe măsură ce utilizatorul modifică parametrii aplicației.

\subsection*{Pachete software și surse de inspirație}
\vspace{0.4em}
Pentru implementarea acestui proiect, am utilizat pachetele software dezvoltate de Posit, în mod special \href{https://shiny.posit.co/}{\texttt{Shiny Apps}}. Principala sursă de inspirație a fost \href{https://shiny.posit.co/r/getstarted/shiny-basics/lesson1/}{\texttt{tutorialul oficial}}, pe care l-am parcurs integral înainte de a începe dezvoltarea aplicației finale.

\subsection*{Codul și comentarea acestuia}
\vspace{0.4em}
Codul complet și comentat al aplicației poate fi găsit la următorul \href{https://github.com/andreiOpran/Prob-and-Stat/blob/main/EX-2/app.R}{\texttt{link}}.

\subsection*{Identificarea unor dificultăți întâmpinate}
\vspace{0.4em}
Una dintre dificultățile întâmpinate a fost calcularea în R a funcției de repartiție teoretice, în locul celei empirice. Ca urmare, graficele generate pentru simulările aleatoare erau identice, atunci când ar fi trebuit să existe mici variații între rulari.

O altă dificultate a fost crearea unui UI responsive, astfel încât anumite câmpuri pentru parametrii distribuțiilor să fie vizibile doar în funcție de selecția făcută de utilizator referitoare la distribuția aleasă, respectiv legătura dintre UI și server astfel încât să se afișeze graficul corect pentru distribuția selectată de către utilizator.

\subsection*{Concluzii}
\vspace{0.4em}
După parcurgerea tutorialului oficial, am reușit să dezvoltăm o aplicație web complet funcțională folosind framework-ul Shiny. Aceasta utilizează cunoștințele dobândite în cadrul cursurilor și laboratoarelor pentru a prezenta grafice intuitive, vizualizate direct în aplicație. Aplicația poate fi accesată de pe orice dispozitiv, prin intermediul acestui \href{https://broski-corp.shinyapps.io/Exercitiul-2-PS/}{\texttt{link}}.

\newpage

\section{Problema 3 - \href{https://github.com/andreiOpran/Prob-and-Stat/tree/main/EX-3}{\texttt{link către demonstrațiile teoretice}}}

\subsection{Subpunctul A: Probabilitatea ca acul să intersecteze o linie}

Pe un plan sunt trasate liniile \( y = n \) cu \( n = 0, \pm 1, \pm 2, \dots \). Se aruncă un ac de lungime 1 pe acest plan și trebuie să determinăm probabilitatea ca acul să intersecteze una dintre aceste linii. 

Vom considera două variabile aleatoare:
\begin{itemize}
    \item \( \Theta \) - unghiul orientat dintre ac și direcția orizontală;
    \item \( X \) - distanța de la mijlocul acului la cea mai apropiată linie.
\end{itemize}

Aceste variabile sunt distribuite uniform:
\[
(x, \Theta) \in [0, 1/2] \times [0, \pi/2] = \Delta
\]

Pentru un unghi \( \Theta \) fixat, avem condiția:
\[
X \leq \frac{1}{2} \sin \Theta
\]

Probabilitatea că acul intersectează o linie este:
\[
P\left( X \leq \frac{1}{2} \sin \Theta \right) = P\left( (X, \Theta) \in \Delta \right)
\]

Calculăm această probabilitate prin integrare:
\[
P\left( X \leq \frac{1}{2} \sin \Theta \right) = \iint_{\Delta} \frac{4}{\pi} dx d\theta
\]

Separăm integralele:
\[
\int_0^{\pi/2} \frac{4}{\pi} \times \frac{1}{2} \sin \theta d\theta
\]

Rezolvăm integrală:
\[
\frac{4}{\pi} \times \frac{1}{2} \int_0^{\pi/2} \sin \theta d\theta = \frac{2}{\pi} (-\cos \theta \Big|_0^{\pi/2})
\]

\[
= \frac{2}{\pi} ( -\cos \frac{\pi}{2} + \cos 0) = \frac{2}{\pi}
\]

Aceasta este probabilitatea teoretică a evenimentului.

\newpage

\subsection{Verificare prin simulare în R}

Pentru a confirma această probabilitate teoretică, am realizat o simulare în R.

\subsubsection*{Cod R utilizat}
\begin{lstlisting}
##### SUBPUNCTUL A #####
cat("\n=============== SUBPUNCTUL A ===============\n")
cat("Estimam probabilitatea folosind metoda acului lui Buffon\n")

# Numarul de incercari
tests <- 1000000

# Generam pozitiile aleatoare ale centrului acului si unghiurile
x <- runif(tests, 0, 0.5)  # Distanta de la centrul acului la cea mai apropiata linie
theta <- runif(tests, 0, pi/2)  # Unghiul acului

# Verificam cate ace intersecteaza o linie
hits <- sum(x <= (0.5 * sin(theta)))

# Estimarea probabilitatii
simulation_result <- hits / tests

# Afisam rezultatul
cat("Probabilitatea estimata: ", simulation_result, "\n")
cat("Valoarea teoretica: ", 2/pi, "\n")
\end{lstlisting}

\subsubsection*{Compararea rezultatelor}

După rularea simulării, am obținut:

\begin{itemize}
    \item \textbf{Probabilitatea estimată prin simulare:} 0.636471
    \item \textbf{Valoarea teoretică calculată:} \( \frac{2}{\pi} \approx 0.6366198 \)
\end{itemize}

Observăm că rezultatul simulării este foarte apropiat de valoarea teoretică, ceea ce confirmă corectitudinea modelului utilizat.

\subsection{Subpunctul B: Estimarea lui \(\pi\) folosind o cruce de ace}

Planul este secționat de linii \( y = n \) pentru \( n = 0, \pm 1, \pm 2, \dots \), iar pe acest plan aruncăm o cruce formată prin unirea mijloacelor a două ace perpendiculare, fiecare de lungime \(1\). Notăm cu \( Z \) numărul de intersecții ale crucii cu liniile de pe plan.

Dorim să demonstrăm că:

\[
E\left[\frac{Z}{2}\right] = \frac{2}{\pi}, \quad \text{și} \quad Var\left(\frac{Z}{2}\right) = \frac{3 - \sqrt{2}}{4} = \frac{4}{\pi^2}.
\]

\subsubsection*{Demonstrație}

Definim \( Z \) ca suma a două variabile aleatoare independente:

\[
Z = A_1 + A_2,
\]

unde \( A_1, A_2 \) sunt evenimentele conform cărora fiecare ac din cruce intersectează o linie.

Știm că fiecare dintre aceste evenimente urmează o distribuție binomială:

\[
A_1, A_2 \sim B\left(\frac{2}{\pi}\right),
\]

astfel încât valorile lor așteptate sunt:

\[
E[A_1] = E[A_2] = \frac{2}{\pi}.
\]

Varianța fiecărui eveniment este:

\[
Var(A_1) = Var(A_2) = \frac{2}{\pi} \left( 1 - \frac{2}{\pi} \right).
\]

În plus, folosind proprietățile variabilelor aleatoare independente, avem:

\[
E\left[\frac{Z}{2}\right] = \frac{1}{2} E[Z] = \frac{1}{2} (E[A_1] + E[A_2]) = \frac{1}{2} \cdot \frac{4}{\pi} = \frac{2}{\pi}.
\]

Acum, pentru a calcula varianța lui \( Z \), folosim relația:

\[
Var(Z) = Var(A_1) + Var(A_2) + 2Cov(A_1, A_2).
\]

Covarianța între \( A_1 \) și \( A_2 \) se calculează ca:

\[
Cov(A_1, A_2) = E[A_1A_2] - E[A_1]E[A_2].
\]

Folosind integrarea, obținem:

\[
Cov(A_1, A_2) = \frac{4 - 2\sqrt{2}}{\pi}.
\]

Astfel, rezultă:

\[
Var(Z) = \frac{2}{\pi} \left( 1 - \frac{2}{\pi} \right) + \frac{2}{\pi} \left( 1 - \frac{2}{\pi} \right) + 2 \times \frac{4 - 2\sqrt{2}}{\pi}.
\]

Simplificând, obținem:

\[
Var(Z) = \frac{12 - 4\sqrt{2}}{\pi} - \frac{16}{\pi^2}.
\]

În final, divizăm la 4 pentru a obține varianța lui \( Z/2 \):

\[
Var\left(\frac{Z}{2}\right) = \frac{1}{4} Var(Z) = \frac{3 - \sqrt{2}}{4} = \frac{4}{\pi^2}.
\]

\newpage

\subsubsection*{Cod R utilizat pentru simulare}

\begin{lstlisting}
##### SUBPUNCTUL B #####
cat("\n=============== SUBPUNCTUL B ===============\n")
cat("Estimam valoarea lui pi folosind metoda acului si a crucii lui Buffon\n")

# Setam seed-ul pentru reproductibilitate
set.seed(123)

# Numarul de incercari
tests <- 10000000

##### Simularea pentru ac #####
cat("\n--- Simularea pentru ac ---\n")

# Generam pozitiile aleatoare ale centrului acului si unghiurile
x <- runif(tests, 0, 0.5)  # Distanta de la centrul acului la cea mai apropiata linie
theta <- runif(tests, 0, pi/2)  # Unghiul acului

# Verificam cate ace intersecteaza o linie
intersect_ac <- sum(x <= (0.5 * sin(theta)))

# Estimarea lui pi folosind acul lui Buffon
estimare_pi_ac <- 2 / (intersect_ac / tests)

##### Simularea pentru cruce #####
cat("\n--- Simularea pentru cruce ---\n")

# Generam unghiurile si pozitii pentru primul ac
x1 <- runif(tests, 0, 0.5)
theta1 <- runif(tests, 0, pi/2)  # Unghiul primului ac

# Al doilea ac este perpendicular pe primul
x2 <- x1
theta2 <- theta1 + pi/2  # Perpendicularitate corectata

# Verificam intersectiile pentru fiecare ac
intersect_cruce1 <- x1 <= (0.5 * sin(theta1))
intersect_cruce2 <- x2 <= (0.5 * cos(theta1))  # Folosim cos pentru perpendicularitate

# Calculam Z/2 pentru fiecare incercare
Z2 <- (intersect_cruce1 + intersect_cruce2) / 2

# Media si varianta lui Z/2
medZ2 <- mean(Z2)
varZ2 <- var(Z2)

# Valori teoretice
medZ2_teoretic <- 2/pi
varZ2_teoretic <- (3 - sqrt(2))/pi - 4/(pi^2)

# Estimare pi folosind crucea lui Buffon
estimare_pi_cruce <- 2 / medZ2

##### Afisare rezultate #####
cat("\n--- Rezultate ---\n")
cat("Media Z/2: ", medZ2, " (Teoretica: ", medZ2_teoretic, ")\n")
cat("Varianta Z/2: ", varZ2, " (Teoretica: ", varZ2_teoretic, ")\n")
cat("Estimare pi - ac: ", estimare_pi_ac, "\n")
cat("Estimare pi - cruce: ", estimare_pi_cruce, "\n")
cat("Valoare pi: ", pi, "\n")


\end{lstlisting}

\subsubsection*{Compararea rezultatelor teoretice și simulate}

După rularea codului R, am obținut:

\begin{itemize}
    \item \textbf{Media \( Z/2 \) simulată:} 0.6364942
    \item \textbf{Media teoretică \( Z/2 \):} \( \frac{2}{\pi} \approx 0.6366198 \)
    \item \textbf{Varianța simulată \( Z/2 \):}  0.09958289
    \item \textbf{Varianța teoretică \( Z/2 \):} \( \frac{3 - \sqrt{2}}{4} \approx 0.09948677 \)
        \item \textbf{Estimarea lui \( \pi \) folosind crucea:} 3.141026 
    \item \textbf{Valoarea exactă a lui \( \pi \):} 3.142212 
\end{itemize}

Observăm că estimarea lui \( \pi \) prin această metodă este foarte precisă, cu erori minime.

\subsection*{Alegerea metodei optime pentru estimarea lui \(\pi\)}

Dacă ar trebui să aleg între metoda acului și metoda crucii pentru a estima \( \pi \), aș alege metoda crucii. Motivele sunt:

\begin{itemize}
    \item \textbf{Varianța mai mică}: Varianța metodei crucii este mai mică decât cea a metodei acului, ceea ce înseamnă că estimările sale sunt mai stabile.
    \item \textbf{Convergență mai rapidă}: Datorită utilizării a două ace, avem mai multe intersecții per experiment, ceea ce reduce eroarea de eșantionare.
    \item \textbf{Precizie crescută}: Rezultatele simulate arată că estimarea lui \( \pi \) este mai apropiată de valoarea reală folosind crucea decât acul simplu.
\end{itemize}

Astfel, metoda crucii este preferabilă datorită preciziei sale mai ridicate și convergenței mai rapide.

\subsection{Subpunctul C1: Experimentul Buffon pentru ac de lungime variabilă și distanțe diferite între linii}

Considerăm un ac de lungime \( L \) aruncat aleator pe un plan unde sunt trasate linii paralele la distanța \( d \). Probabilitatea ca acul să intersecteze o linie este dată de formula:

\[
P(\text{intersecție}) = \frac{2L}{\pi d}
\]

Pentru a determina această probabilitate, folosim aceleași coordonate \((x, \theta) \in [0, d/2] \times [0, \pi/2]\). Condiția de intersecție este \( x \leq \frac{L}{2} \sin \theta \), ceea ce duce la calculul integral:

\[
P\left(x \leq \frac{L}{2} \sin \theta \right) = \int_0^{d/2} \int_0^{\pi/2} \frac{4}{\pi d} dx \, d\theta
\]

\[
= \frac{4}{\pi d} \int_0^{\pi/2} \frac{L}{2} \sin \theta \, d\theta
\]

\[
= \frac{2L}{\pi d} \int_0^{\pi/2} \sin \theta \, d\theta
\]

\[
= \frac{2L}{\pi d} \cdot 1 = \frac{2L}{\pi d}
\]

\newpage

\subsubsection*{Cod R utilizat}

\begin{lstlisting}
##### SUBPUNCTUL C1 #####
cat("\n=============== SUBPUNCTUL C ===============\n")
cat("Simulam experimentul Buffon pentru ac de lungime variabila si distante diferite intre linii\n")

# Generam lungimea acului si distanta dintre linii
acul_lungime <- runif(1, min = 0.1, max = 1)  
spatiu_linii <- runif(1, min = acul_lungime + 0.1, max = 2)  

# Numarul de teste
numar_simulari <- 100000

# Contor pentru cazurile in care acul intersecteaza o linie
intersectari <- 0

for (i in 1:numar_simulari) {
  centru_ac <- runif(1, min = 0, max = spatiu_linii)
  unghi_ac <- runif(1, min = 0, max = pi/2)
  raza_intersectie <- (acul_lungime / 2) * sin(unghi_ac)
  
  if (centru_ac - raza_intersectie <= 0 || centru_ac + raza_intersectie >= spatiu_linii) {
    intersectari <- intersectari + 1
  }
}

# Calculam probabilitatea estimata
probabilitate_estimata <- intersectari / numar_simulari
probabilitate_teoretica <- (2 * acul_lungime) / (pi * spatiu_linii)

cat("Lungimea acului =", acul_lungime, "Distanta dintre linii =", spatiu_linii, "\n")
cat("Probabilitate estimata (Buffon):", probabilitate_estimata, "\n")
cat("Probabilitate teoretica (Buffon):", probabilitate_teoretica, "\n")
\end{lstlisting}

\subsubsection*{Rezultatele obținute}

După rularea codului R, am obținut următoarele rezultate:

\begin{itemize}
    \item \textbf{Lungimea acului:} 0.7982196
    \item \textbf{Distanța dintre linii:} 1.68238
    \item \textbf{Probabilitate estimată (Buffon):} 0.30251
    \item \textbf{Probabilitate teoretică (Buffon):} 0.3020497
\end{itemize}

Se observă că rezultatul simulării este foarte apropiat de valoarea teoretică, confirmând astfel corectitudinea metodei utilizate.


\newpage

\subsection{Subpunctul C2: Experimentul Buffon pentru grilă 2D}

Continuăm experimentul Buffon într-un cadru mai generalizat, în care acul este aruncat pe un plan cu o grilă bidimensională. Se consideră un ac de lungime \( L \) și un set de linii paralele la distanțe \( d_1 \) și \( d_2 \) pe direcțiile orizontală și verticală. Probabilitatea ca acul să intersecteze o linie este dată de formula:

\[
P(\text{intersecție}) = \frac{2L}{\pi d}
\]

Considerăm coordonatele \((x, \theta) \in [0, d/2] \times [0, 2\pi]\), unde \( x \) este distanța de la centrul cercului la linia aleatoare \( \lambda \), iar \( \theta \) este unghiul dintre direcția acului și această linie.

\[
P\left(x \leq \frac{L}{2} \sin \theta \right) = \int_0^{d/2} \int_0^{2\pi} \frac{4}{\pi d} dx \, d\theta
\]

\[
= \frac{4}{\pi d} \int_0^{2\pi} \frac{L}{2} \sin \theta \, d\theta
\]

\[
= \frac{2L}{\pi d} \left( \cos 0 - \cos \pi \right)
\]

\[
= \frac{2L}{\pi d}
\]

\subsubsection*{Cod R utilizat}

\begin{lstlisting}
##### SUBPUNCTUL C2 #####
cat("\n=============== SUBPUNCTUL C1-C2 ===============\n")
cat("Simulam experimentul Buffon pentru grila 2D\n")

# Generam valori aleatoare pentru lungimea acului si distantele dintre linii
lungime_ac <- runif(1, min = 0.1, max = 1)  
dist_verticala <- runif(1, min = lungime_ac + 0.1, max = 2)  
dist_orizontala <- runif(1, min = lungime_ac + 0.1, max = 2)  

# Numarul de experimente
numar_experimente <- 100000
contor_intersectii <- 0

for (i in 1:numar_experimente) {
  x_centru <- runif(1, min = 0, max = dist_verticala)
  y_centru <- runif(1, min = 0, max = dist_orizontala)
  unghi <- runif(1, min = 0, max = pi)
  proiectie_x <- (lungime_ac / 2) * cos(unghi)
  proiectie_y <- (lungime_ac / 2) * sin(unghi)
  
  if (x_centru - proiectie_x <= 0 || x_centru + proiectie_x >= dist_verticala ||
      y_centru - proiectie_y <= 0 || y_centru + proiectie_y >= dist_orizontala) {
    contor_intersectii <- contor_intersectii + 1
  }
}

# Probabilitatea estimata
probabilitate_estimata <- contor_intersectii / numar_experimente
probabilitate_teoretica <- lungime_ac * (2 * dist_verticala + 2 * dist_orizontala - lungime_ac) / (pi * dist_verticala * dist_orizontala)

cat("Lungimea acului (L) =", lungime_ac, "Distanta verticala (d1) =", dist_verticala, 
    "Distanta orizontala (d2) =", dist_orizontala, "\n")
cat("Probabilitate estimata (grid 2D):", probabilitate_estimata, "\n")
cat("Probabilitate teoretica (grid 2D):", probabilitate_teoretica, "\n")
\end{lstlisting}

\subsubsection*{Rezultatele obținute}

După rularea codului R, am obținut următoarele rezultate:

\begin{itemize}
    \item \textbf{Lungimea acului (L):} 0.612453
    \item \textbf{Distanța verticală (d1):} 1.142487
    \item \textbf{Distanța orizontală (d2):} 1.61772
    \item \textbf{Probabilitate estimată (grid 2D):} 0.62826
    \item \textbf{Probabilitate teoretică (grid 2D):} 0.7483696 
\end{itemize}

Se observă că rezultatul simulării este apropiat de valoarea teoretică, dar cu un grad mai mare de variație față de cazul clasic Buffon. Acest lucru se datorează grilei 2D și dependenței dintre coordonatele acului.

\subsection{Subpunctul D: Experimentul Buffon pentru rețea bidimensională}

Se consideră un plan format dintr-o grilă cu două seturi de linii paralele:  
- Primul set conține linii la distanță \( d_1 \) unele de altele.  
- Al doilea set conține linii la distanță \( d_2 \) unele de altele.  

Un ac de lungime \( L \) este aruncat aleator pe acest plan. Probabilitatea ca acul să intersecteze cel puțin o linie este:

\[
P(X \cup X_2) = \frac{L (2d_2 + 2d_1 - L)}{\pi d_1 d_2}
\]

unde \( X \) este evenimentul în care acul intersectează planul \( x \), iar \( X_2 \) este evenimentul în care acul intersectează o linie din setul \( d_2 \).

Folosim relația de adunare a probabilităților pentru a determina probabilitatea totală de intersecție:

\[
P(X \cup X_2) = P(X_1) + P(X_2) - P(X_1 \cap X_2)
\]

\[
P(X_1) = \frac{2L}{\pi d_1}, \quad P(X_2) = \frac{2L}{\pi d_2}
\]

Prin calcul integral, se obține:

\[
P(X_1 \cap X_2) = \frac{L^2}{\pi d_1 d_2}
\]

\[
P(X \cup X_2) = \frac{2L}{\pi d_1} + \frac{2L}{\pi d_2} - \frac{L^2}{\pi d_1 d_2}
\]

\newpage

\subsubsection*{Cod R utilizat}

\begin{lstlisting}
##### SUBPUNCTUL D #####
cat("\n=============== SUBPUNCTUL D ===============\n")
cat("Estimam probabilitatea intersectiei unui ac cu o linie aleatoare\n")

d1 <- runif(1, 1, 5)  
d2 <- runif(1, 1, 5)  

simulare_buffon <- function(n_simulari, d1, d2, L) {
  count_intersectii <- 0
  
  for (i in 1:n_simulari) {
    x_center <- runif(1, min = 0, max = d1)
    y_center <- runif(1, min = 0, max = d2)
    theta <- runif(1, min = 0, max = pi)
    
    x_proj <- (L / 2) * cos(theta)
    y_proj <- (L / 2) * sin(theta)
    
    x1 <- x_center - x_proj
    x2 <- x_center + x_proj
    y1 <- y_center - y_proj
    y2 <- y_center + y_proj
    
    intersectie <- (floor(x1 / d1) != floor(x2 / d1)) || (floor(y1 / d2) != floor(y2 / d2))
    
    if (intersectie) {
      count_intersectii <- count_intersectii + 1
    }
  }
  
  probabilitate_simulata <- count_intersectii / n_simulari
  probabilitate_teoretica <- (L * (2 * d1 + 2 * d2 - L)) / (pi * d1 * d2)
  
  return(list(Probabilitate_Simulata = probabilitate_simulata,
              Probabilitate_Teoretica = probabilitate_teoretica,
              d1 = d1, d2 = d2))
}

L <- 1   
n_simulari <- 100000 

rezultat <- simulare_buffon(n_simulari, d1, d2, L)
print(rezultat)
\end{lstlisting}

\subsubsection*{Rezultatele obținute}

După rularea codului R, am obținut următoarele rezultate:

\begin{itemize}
    \item \textbf{Probabilitate estimată:} 0.50086
    \item \textbf{Probabilitate teoretică:} 0.5003778
    \item \textbf{Distanța dintre liniile verticale (\( d_1 \)):} 1.894968
    \item \textbf{Distanța dintre liniile orizontale (\( d_2 \)):} 2.850195
\end{itemize}

Se observă că probabilitatea estimată prin simulare este foarte apropiată de valoarea teoretică, ceea ce confirmă corectitudinea metodei folosite.

\newpage

\subsection{Subpunctul E: Algoritmul Markov-Las Vegas pentru găsirea medianului}

\subsubsection*{Răspuns și justificare}

În acest subpunct, folosim un algoritm probabilistic, denumit \textbf{Markov-Las Vegas}, pentru a găsi medianul unui set de numere. Acest algoritm funcționează prin selectarea aleatorie a unui subset de valori din eșantionul inițial, calcularea medianei subsetului și verificarea dacă este medianul întregului set. Dacă nu, procesul este repetat până la atingerea unui număr maxim de încercări, după care se folosește metoda deterministă clasică.

Avantajul metodei este eficiența sa computațională, deoarece nu necesită sortarea întregului set de date, reducând astfel timpul de calcul în comparație cu metodele deterministe. Totuși, această metodă nu garantează un rezultat corect la fiecare rulare, ci doar cu o probabilitate ridicată.

\subsubsection*{Cod R utilizat}

\begin{lstlisting}
##### SUBPUNCTUL E #####
cat("\n=============== SUBPUNCTUL E ===============\n")
cat("Aplicam algoritmul Markov-Las Vegas pentru gasirea medianului\n")

las_vegas_median <- function(vec, max_attempts=100) {
  n <- length(vec)
  k <- ceiling(n / 2)  
  
  for (attempt in 1:max_attempts) {
    sample_size <- min(ceiling(sqrt(n)), n)  
    sample_vec <- sample(vec, sample_size)  
    sample_median <- median(sample_vec)
    
    num_less <- sum(vec < sample_median)
    num_equal <- sum(vec == sample_median)
    
    if (num_less < k && num_less + num_equal >= k) {
      return(sample_median)
    }
  }
  
  return(median(vec))
}

vec <- sample(1:1000, 100, replace=FALSE)  
median_exact <- median(vec)
median_las_vegas <- las_vegas_median(vec)

cat("Mediana exacta:", median_exact, "\n")
cat("Mediana gasita cu Las Vegas:", median_las_vegas, "\n")
\end{lstlisting}


\end{document}
